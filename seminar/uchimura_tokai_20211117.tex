%%#! latex

%%%%%%%%%%             Preamble                   %%%%%%%%%%

\documentclass[12pt]{jarticle}
\usepackage{amsfonts,amssymb,amsmath,amscd,amsthm,latexsym}
\usepackage{graphicx}
\usepackage{fancybox}
\usepackage{okumacro}
\usepackage{ulem}
%\usepackage{color}
%\usepackage{hyperref}
%\usepackage{showkeys}
%\usepackage{fancybox,ascmac}
%\usepackage{fancyhdr}
\pagestyle{empty}
\usepackage[dvipdfmx]{hyperref}

%%%%%%%%%%             Layout                     %%%%%%%%%%

\setlength{\topmargin}{-1cm}
\setlength{\textheight}{25cm}
\setlength{\textwidth}{16.5cm}
%\setlength{\oddsidemargin}{0cm}
%\setlength{\evensidemargin}{0cm}
%\addtolength{\textwidth}{.815in}
\addtolength{\oddsidemargin}{-2cm}
\addtolength{\evensidemargin}{-2cm}

%%%%%%%%%%           Definition, Theorem, etc.     %%%%%%%%%%

\newtheorem{thm}{定理}
\newtheorem{lem}[thm]{補題}
\newtheorem{prop}[thm]{命題}
\newtheorem{claim}[thm]{主張}
\newtheorem{cor}[thm]{系}
\newtheorem{prob}[thm]{問題}
\newtheorem{conj}[thm]{予想}
\theoremstyle{definition}
\newtheorem{dfn}[thm]{定義} 
\newtheorem{ex}[thm]{例}
\newtheorem{reidai}[thm]{例題}
\newtheorem{rmk}[thm]{注意}

%\numberwithin{thm}{section}
%\numberwithin{equation}{section}

%%%%%%%%%%             Macro (by Nasu)           %%%%%%%%%%

\newcommand{\Proof}{\noindent {\bf 証明)}\ \ }
\newcommand{\Hom}{\operatorname{Hom}}
\newcommand{\Ext}{\operatorname{Ext}}
\newcommand{\Spec}{\operatorname{Spec}}
\newcommand{\Proj}{\operatorname{Proj}}
\newcommand{\ob}{\operatorname{ob}}
\newcommand{\Hilb}{\operatorname{Hilb}}
\newcommand{\red}{\operatorname{red}}
\newcommand{\im}{\operatorname{im}}
\newcommand{\Bs}{\operatorname{Bs}}
\newcommand{\Jac}{\operatorname{Jac}}
\newcommand{\Pic}{\operatorname{Pic}}
\newcommand{\coker}{\operatorname{coker}}
\newcommand{\Gr}{\operatorname{Gr}}
\newcommand{\Bl}{\operatorname{Bl}}
%\newcommand{\algeeq}{\stackrel{alg.}{\; \sim \;}}
\newcommand{\Rat}{\operatorname{Rat}}
\newcommand{\Exc}{\operatorname{Exc}}
\newcommand{\car}{\operatorname{char}}
\newcommand{\Aut}{\operatorname{Aut}}
\newcommand{\sat}{\operatorname{sat}}
\newcommand{\pd}{\operatorname{dp}}
\newcommand{\rank}{\operatorname{rank}}
\newcommand{\Ann}{\operatorname{Ann}}

\renewcommand{\theenumi}{\arabic{enumi}}
\renewcommand{\theenumii}{\alph{enumii}}
\renewcommand{\theenumiii}{\roman{enumiii}}
\renewcommand{\labelenumi}{(\theenumi)}
\renewcommand{\labelenumii}{(\theenumii)}
\renewcommand{\labelenumiii}{[\theenumiii]}
\renewcommand{\baselinestretch}{1.2}
%\renewcommand{\abstractname}{\sc Abstract}

%%%%%%%%%%           Commutative diagram          %%%%%%%%%%

% A litte smaller diagram than AMS CD envrionment

\newcommand{\mapright}[1]{%
\smash{\mathop{%
\hbox to 1cm{\rightarrowfill}}\limits^{#1}}}
\newcommand{\mapleft}[1]{%
\smash{\mathop{%
\hbox to 1cm{\leftarrowfill}}\limits^{#1}}}
\newcommand{\mapdown}[1]{\Big\downarrow
\llap{$\vcenter{\hbox{$\scriptstyle#1\,$}}$ }}
\newcommand{\mapup}[1]{\Big\uparrow
\rlap{$\vcenter{\hbox{$\scriptstyle#1$}}$ }}

%%%%%%%%%%              Title                    %%%%%%%%%%

\title{}
\author{}
\date{}

%%%%%%%%%%            Text Start                 %%%%%%%%%%

\begin{document}

%\maketitle

\begin{center}
 {\bf \Large 東海大学理学部数学・情報数理談話会(Zoom)}
\end{center}

以下の要領において談話会を開催致します. 多数の方の御来聴を
お待ち致しております. 

\smallskip

\begin{center}
  \begin{tabular}{rl}
    日程 & 2021年11月17日(水) 17:15 〜 18:15 \\
    場所 & Zoom\\
    招待リンク & \href{https://us02web.zoom.us/j/89409720082}{https://us02web.zoom.us/j/89409720082} \\
    & (パスコードについては案内メールをご参照ください) \\
    講演者 & 内村桂輔氏 (東海大学名誉教授)\\
    タイトル & Dynamics of Chebyshev endomorphisms on some affine algebraic varieties
  \end{tabular}
\end{center}

\noindent
{\bf アブストラクト}:\quad
The Chebyshev polynomials \(T_d\) in one variable are typical chaotic maps on \({\mathbb C}\). Chebyshev endomorphisms \(P_{A_n}^{d} : {\mathbb C}^n \to {\mathbb C}^n\) \ are also chaotic. We consider the action of the dihedral group \(D_{n+1}\) \ on \({\mathbb C}^n\). The endomorphism \(P_{A_n}^{d}\) maps any \(D_{n+1}-\)orbit of \({\bf z} \in {\mathbb C}^n\) to a \ \(D_{n+1}\)-orbit of \(P_{A_n}^{d}({\bf z})\). The endomorphism \(P_{A_n}^{d}\) induces a mapping on \ \({\mathbb C}^n/D_{n+1}\). Using invariant theory we embed \ \({\mathbb C}^n/D_{n+1}\) \ as an affine subvariety \(X\) in \({\mathbb C}^m\). Then we have morphisms \(g_d\) on \(X\). We study the cases \ \(n = 2\) \ and \(3\). In these cases the morphisms \(g_d\) are defined over \({\mathbb Z}\). We find a class of affine subvarieties \(V\) of \(X\) which are invariant under \(g_d\). These varieties are concerned with branch loci or critical loci.The class contains \ \({\mathbb C}^2\), a cuspidal cubic, a parabola, a quadric hypersurface in \({\mathbb C}^4\), an affine algebraic surface in \({\mathbb C}^4\) which is birationally equivalent to an affine quadric cone in \({\mathbb C}^3\), and others. For each affine variety \(V\) in the class, there exists a polynomial parametrization \(P_V\) satisfying \ \(g_d\mid_V(P_V(y_1, \dots , y_k)) = P_V(T_d(y_1), \dots , T_d(y_k))\), \ where \ \(T_d(z)\) \ is a Chebyshev polynomial in one variable. Then we determine the set of bounded orbits of \ \(g_d\mid_V\) in each invariant set \(V\) and give relations between them. 

\vskip 2cm

\begin{flushright}
  \begin{tabular}{rl}
      世話人: & 那須弘和(情報数理学科)\\
      & nasu@tokai-u.jp 
  \end{tabular}

\end{flushright}

\end{document}

